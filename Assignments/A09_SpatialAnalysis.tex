% Options for packages loaded elsewhere
\PassOptionsToPackage{unicode}{hyperref}
\PassOptionsToPackage{hyphens}{url}
%
\documentclass[
]{article}
\usepackage{amsmath,amssymb}
\usepackage{lmodern}
\usepackage{ifxetex,ifluatex}
\ifnum 0\ifxetex 1\fi\ifluatex 1\fi=0 % if pdftex
  \usepackage[T1]{fontenc}
  \usepackage[utf8]{inputenc}
  \usepackage{textcomp} % provide euro and other symbols
\else % if luatex or xetex
  \usepackage{unicode-math}
  \defaultfontfeatures{Scale=MatchLowercase}
  \defaultfontfeatures[\rmfamily]{Ligatures=TeX,Scale=1}
\fi
% Use upquote if available, for straight quotes in verbatim environments
\IfFileExists{upquote.sty}{\usepackage{upquote}}{}
\IfFileExists{microtype.sty}{% use microtype if available
  \usepackage[]{microtype}
  \UseMicrotypeSet[protrusion]{basicmath} % disable protrusion for tt fonts
}{}
\makeatletter
\@ifundefined{KOMAClassName}{% if non-KOMA class
  \IfFileExists{parskip.sty}{%
    \usepackage{parskip}
  }{% else
    \setlength{\parindent}{0pt}
    \setlength{\parskip}{6pt plus 2pt minus 1pt}}
}{% if KOMA class
  \KOMAoptions{parskip=half}}
\makeatother
\usepackage{xcolor}
\IfFileExists{xurl.sty}{\usepackage{xurl}}{} % add URL line breaks if available
\IfFileExists{bookmark.sty}{\usepackage{bookmark}}{\usepackage{hyperref}}
\hypersetup{
  pdftitle={Assignment 9: Spatial Analysis in R},
  pdfauthor={Nancy Bao},
  hidelinks,
  pdfcreator={LaTeX via pandoc}}
\urlstyle{same} % disable monospaced font for URLs
\usepackage[margin=2.54cm]{geometry}
\usepackage{color}
\usepackage{fancyvrb}
\newcommand{\VerbBar}{|}
\newcommand{\VERB}{\Verb[commandchars=\\\{\}]}
\DefineVerbatimEnvironment{Highlighting}{Verbatim}{commandchars=\\\{\}}
% Add ',fontsize=\small' for more characters per line
\usepackage{framed}
\definecolor{shadecolor}{RGB}{248,248,248}
\newenvironment{Shaded}{\begin{snugshade}}{\end{snugshade}}
\newcommand{\AlertTok}[1]{\textcolor[rgb]{0.94,0.16,0.16}{#1}}
\newcommand{\AnnotationTok}[1]{\textcolor[rgb]{0.56,0.35,0.01}{\textbf{\textit{#1}}}}
\newcommand{\AttributeTok}[1]{\textcolor[rgb]{0.77,0.63,0.00}{#1}}
\newcommand{\BaseNTok}[1]{\textcolor[rgb]{0.00,0.00,0.81}{#1}}
\newcommand{\BuiltInTok}[1]{#1}
\newcommand{\CharTok}[1]{\textcolor[rgb]{0.31,0.60,0.02}{#1}}
\newcommand{\CommentTok}[1]{\textcolor[rgb]{0.56,0.35,0.01}{\textit{#1}}}
\newcommand{\CommentVarTok}[1]{\textcolor[rgb]{0.56,0.35,0.01}{\textbf{\textit{#1}}}}
\newcommand{\ConstantTok}[1]{\textcolor[rgb]{0.00,0.00,0.00}{#1}}
\newcommand{\ControlFlowTok}[1]{\textcolor[rgb]{0.13,0.29,0.53}{\textbf{#1}}}
\newcommand{\DataTypeTok}[1]{\textcolor[rgb]{0.13,0.29,0.53}{#1}}
\newcommand{\DecValTok}[1]{\textcolor[rgb]{0.00,0.00,0.81}{#1}}
\newcommand{\DocumentationTok}[1]{\textcolor[rgb]{0.56,0.35,0.01}{\textbf{\textit{#1}}}}
\newcommand{\ErrorTok}[1]{\textcolor[rgb]{0.64,0.00,0.00}{\textbf{#1}}}
\newcommand{\ExtensionTok}[1]{#1}
\newcommand{\FloatTok}[1]{\textcolor[rgb]{0.00,0.00,0.81}{#1}}
\newcommand{\FunctionTok}[1]{\textcolor[rgb]{0.00,0.00,0.00}{#1}}
\newcommand{\ImportTok}[1]{#1}
\newcommand{\InformationTok}[1]{\textcolor[rgb]{0.56,0.35,0.01}{\textbf{\textit{#1}}}}
\newcommand{\KeywordTok}[1]{\textcolor[rgb]{0.13,0.29,0.53}{\textbf{#1}}}
\newcommand{\NormalTok}[1]{#1}
\newcommand{\OperatorTok}[1]{\textcolor[rgb]{0.81,0.36,0.00}{\textbf{#1}}}
\newcommand{\OtherTok}[1]{\textcolor[rgb]{0.56,0.35,0.01}{#1}}
\newcommand{\PreprocessorTok}[1]{\textcolor[rgb]{0.56,0.35,0.01}{\textit{#1}}}
\newcommand{\RegionMarkerTok}[1]{#1}
\newcommand{\SpecialCharTok}[1]{\textcolor[rgb]{0.00,0.00,0.00}{#1}}
\newcommand{\SpecialStringTok}[1]{\textcolor[rgb]{0.31,0.60,0.02}{#1}}
\newcommand{\StringTok}[1]{\textcolor[rgb]{0.31,0.60,0.02}{#1}}
\newcommand{\VariableTok}[1]{\textcolor[rgb]{0.00,0.00,0.00}{#1}}
\newcommand{\VerbatimStringTok}[1]{\textcolor[rgb]{0.31,0.60,0.02}{#1}}
\newcommand{\WarningTok}[1]{\textcolor[rgb]{0.56,0.35,0.01}{\textbf{\textit{#1}}}}
\usepackage{graphicx}
\makeatletter
\def\maxwidth{\ifdim\Gin@nat@width>\linewidth\linewidth\else\Gin@nat@width\fi}
\def\maxheight{\ifdim\Gin@nat@height>\textheight\textheight\else\Gin@nat@height\fi}
\makeatother
% Scale images if necessary, so that they will not overflow the page
% margins by default, and it is still possible to overwrite the defaults
% using explicit options in \includegraphics[width, height, ...]{}
\setkeys{Gin}{width=\maxwidth,height=\maxheight,keepaspectratio}
% Set default figure placement to htbp
\makeatletter
\def\fps@figure{htbp}
\makeatother
\setlength{\emergencystretch}{3em} % prevent overfull lines
\providecommand{\tightlist}{%
  \setlength{\itemsep}{0pt}\setlength{\parskip}{0pt}}
\setcounter{secnumdepth}{-\maxdimen} % remove section numbering
\ifluatex
  \usepackage{selnolig}  % disable illegal ligatures
\fi

\title{Assignment 9: Spatial Analysis in R}
\author{Nancy Bao}
\date{}

\begin{document}
\maketitle

\hypertarget{overview}{%
\subsection{OVERVIEW}\label{overview}}

This exercise accompanies the lessons in Environmental Data Analytics
(ENV872L) on spatial analysis.

\hypertarget{directions}{%
\subsubsection{Directions}\label{directions}}

\begin{enumerate}
\def\labelenumi{\arabic{enumi}.}
\tightlist
\item
  Change ``Student Name'' on line 3 (above) with your name.
\item
  Use the lesson as a guide. It contains code that can be modified to
  complete the assignment.
\item
  Work through the steps, \textbf{creating code and output} that fulfill
  each instruction.
\item
  Be sure to \textbf{answer the questions} in this assignment document.
  Space for your answers is provided in this document and is indicated
  by the ``\textgreater{}'' character. If you need a second paragraph be
  sure to start the first line with ``\textgreater{}''. You should
  notice that the answer is highlighted in green by RStudio.
\item
  When you have completed the assignment, \textbf{Knit} the text and
  code into a single HTML file.
\item
  After Knitting, please submit the completed exercise (PDF file) in
  Sakai. Please add your last name into the file name (e.g.,
  ``Fay\_A10\_SpatialAnalysis.pdf'') prior to submission.
\end{enumerate}

\hypertarget{data-wrangling}{%
\subsection{DATA WRANGLING}\label{data-wrangling}}

\hypertarget{set-up-your-session}{%
\subsubsection{Set up your session}\label{set-up-your-session}}

\begin{enumerate}
\def\labelenumi{\arabic{enumi}.}
\tightlist
\item
  Check your working directory
\item
  Import libraries: tidyverse, sf, leaflet, and mapview
\end{enumerate}

\begin{Shaded}
\begin{Highlighting}[]
\CommentTok{\#1. Check working directory}
\FunctionTok{getwd}\NormalTok{()}
\end{Highlighting}
\end{Shaded}

\begin{verbatim}
## [1] "/Users/Nancy/Desktop/Semester 4/ENV 872L/Environmental_Data_Analytics_2021"
\end{verbatim}

\begin{Shaded}
\begin{Highlighting}[]
\CommentTok{\#2. Import libraries}
\FunctionTok{library}\NormalTok{(tidyverse)}
\end{Highlighting}
\end{Shaded}

\begin{verbatim}
## -- Attaching packages -------------------------------------------------------------------------------------------- tidyverse 1.3.0 --
\end{verbatim}

\begin{verbatim}
## v ggplot2 3.3.3     v purrr   0.3.4
## v tibble  3.0.4     v dplyr   1.0.5
## v tidyr   1.1.3     v stringr 1.4.0
## v readr   1.3.1     v forcats 0.5.0
\end{verbatim}

\begin{verbatim}
## -- Conflicts ----------------------------------------------------------------------------------------------- tidyverse_conflicts() --
## x dplyr::filter() masks stats::filter()
## x dplyr::lag()    masks stats::lag()
\end{verbatim}

\begin{Shaded}
\begin{Highlighting}[]
\FunctionTok{library}\NormalTok{(sf)}
\end{Highlighting}
\end{Shaded}

\begin{verbatim}
## Linking to GEOS 3.8.1, GDAL 3.1.4, PROJ 6.3.1
\end{verbatim}

\begin{Shaded}
\begin{Highlighting}[]
\FunctionTok{library}\NormalTok{(leaflet)}
\FunctionTok{library}\NormalTok{(mapview)}
\end{Highlighting}
\end{Shaded}

\begin{verbatim}
## GDAL version >= 3.1.0 | setting mapviewOptions(fgb = TRUE)
\end{verbatim}

\hypertarget{read-and-filter-county-features-into-an-sf-dataframe-and-plot}{%
\subsubsection{Read (and filter) county features into an sf dataframe
and
plot}\label{read-and-filter-county-features-into-an-sf-dataframe-and-plot}}

In this exercise, we will be exploring stream gage height data in
Nebraska corresponding to floods occurring there in 2019. First, we will
import from the US Counties shapefile we've used in lab lessons,
filtering it this time for just Nebraska counties. Nebraska's state FIPS
code is \texttt{31} (as North Carolina's was \texttt{37}).

\begin{enumerate}
\def\labelenumi{\arabic{enumi}.}
\setcounter{enumi}{2}
\tightlist
\item
  Read the \texttt{cb\_2018\_us\_county\_20m.shp} shapefile into an sf
  dataframe, filtering records for Nebraska counties (State FIPS = 31)
\item
  Reveal the dataset's coordinate reference system
\item
  Plot the records as a map (using \texttt{mapview} or \texttt{ggplot})
\end{enumerate}

\begin{Shaded}
\begin{Highlighting}[]
\CommentTok{\#3. Read in Counties shapefile into an sf dataframe, filtering for just NE counties}
\NormalTok{NE\_counties\_sf }\OtherTok{\textless{}{-}} \FunctionTok{st\_read}\NormalTok{(}\StringTok{"./Data/Spatial/cb\_2018\_us\_county\_20m.shp"}\NormalTok{) }\SpecialCharTok{\%\textgreater{}\%}
                  \FunctionTok{filter}\NormalTok{(STATEFP }\SpecialCharTok{==} \DecValTok{31}\NormalTok{)}
\end{Highlighting}
\end{Shaded}

\begin{verbatim}
## Reading layer `cb_2018_us_county_20m' from data source `/Users/Nancy/Desktop/Semester 4/ENV 872L/Environmental_Data_Analytics_2021/Data/Spatial/cb_2018_us_county_20m.shp' using driver `ESRI Shapefile'
## Simple feature collection with 3220 features and 9 fields
## Geometry type: MULTIPOLYGON
## Dimension:     XY
## Bounding box:  xmin: -179.1743 ymin: 17.91377 xmax: 179.7739 ymax: 71.35256
## Geodetic CRS:  NAD83
\end{verbatim}

\begin{Shaded}
\begin{Highlighting}[]
\CommentTok{\#4. Reveal the CRS of the counties features}
\FunctionTok{st\_crs}\NormalTok{(NE\_counties\_sf)}
\end{Highlighting}
\end{Shaded}

\begin{verbatim}
## Coordinate Reference System:
##   User input: NAD83 
##   wkt:
## GEOGCRS["NAD83",
##     DATUM["North American Datum 1983",
##         ELLIPSOID["GRS 1980",6378137,298.257222101,
##             LENGTHUNIT["metre",1]]],
##     PRIMEM["Greenwich",0,
##         ANGLEUNIT["degree",0.0174532925199433]],
##     CS[ellipsoidal,2],
##         AXIS["latitude",north,
##             ORDER[1],
##             ANGLEUNIT["degree",0.0174532925199433]],
##         AXIS["longitude",east,
##             ORDER[2],
##             ANGLEUNIT["degree",0.0174532925199433]],
##     ID["EPSG",4269]]
\end{verbatim}

\begin{Shaded}
\begin{Highlighting}[]
\FunctionTok{st\_crs}\NormalTok{(NE\_counties\_sf)}\SpecialCharTok{$}\NormalTok{epsg}
\end{Highlighting}
\end{Shaded}

\begin{verbatim}
## [1] 4269
\end{verbatim}

\begin{Shaded}
\begin{Highlighting}[]
\CommentTok{\#The CRS is NAD83 (North American Datum 1983); epsg code: 4269}

\CommentTok{\#5. Plot the data}
\NormalTok{NEcounties\_plot}\OtherTok{\textless{}{-}}\FunctionTok{ggplot}\NormalTok{()}\SpecialCharTok{+}
                \FunctionTok{geom\_sf}\NormalTok{(}\AttributeTok{data=}\NormalTok{NE\_counties\_sf, }\FunctionTok{aes}\NormalTok{(}\AttributeTok{color=}\StringTok{"NAME"}\NormalTok{),}\AttributeTok{fill=}\StringTok{"white"}\NormalTok{)}\SpecialCharTok{+}
                \FunctionTok{scale\_color\_discrete}\NormalTok{(}\AttributeTok{name=}\ConstantTok{NULL}\NormalTok{,}\AttributeTok{labels=}\FunctionTok{c}\NormalTok{(}\StringTok{"Nebraska counties"}\NormalTok{))}
\FunctionTok{print}\NormalTok{(NEcounties\_plot)}
\end{Highlighting}
\end{Shaded}

\includegraphics{A09_SpatialAnalysis_files/figure-latex/Read the county data into an sf dataframe-1.pdf}

\begin{Shaded}
\begin{Highlighting}[]
\CommentTok{\#I used scale\_color\_discrete to rename the legend to "Nebraska counties".}
\end{Highlighting}
\end{Shaded}

\begin{enumerate}
\def\labelenumi{\arabic{enumi}.}
\setcounter{enumi}{5}
\tightlist
\item
  What is the EPSG code of the Counties dataset? Is this a geographic or
  a projected coordinate reference system? (Or, does this CRS use
  angular or planar coordinate units?) To what datum is this CRS
  associated? (Tip: look the EPSG code on
  \url{https://spatialreference.org})
\end{enumerate}

\begin{quote}
ANSWER: The EPSG code of the Counties dataset is 4269. This is a
geographic coordinate reference system. This CRS uses angular coordinate
units (longitude/latitude). The CRS is associated with the North
American Datum 1983.
\end{quote}

\hypertarget{read-in-gage-locations-csv-as-a-dataframe-then-display-the-column-names-it-contains}{%
\subsubsection{Read in gage locations csv as a dataframe, then display
the column names it
contains}\label{read-in-gage-locations-csv-as-a-dataframe-then-display-the-column-names-it-contains}}

Next we'll read in some USGS/NWIS gage location data added to the
\texttt{Data/Raw} folder. These are in the
\texttt{NWIS\_SiteInfo\_NE\_RAW.csv} file.(See
\texttt{NWIS\_SiteInfo\_NE\_RAW.README.txt} for more info on this
dataset.)

\begin{enumerate}
\def\labelenumi{\arabic{enumi}.}
\setcounter{enumi}{6}
\item
  Read the NWIS\_SiteInfo\_NE\_RAW.csv file into a standard dataframe.
\item
  Display the column names of this dataset.
\end{enumerate}

\begin{Shaded}
\begin{Highlighting}[]
\CommentTok{\#7. Read in gage locations csv as a dataframe}
\NormalTok{usgs\_gage\_locations }\OtherTok{\textless{}{-}}\FunctionTok{read.csv}\NormalTok{(}\StringTok{"./Data/Raw/NWIS\_SiteInfo\_NE\_RAW.csv"}\NormalTok{)}

\CommentTok{\#8. Reveal the names of the columns}
\FunctionTok{colnames}\NormalTok{(usgs\_gage\_locations)}
\end{Highlighting}
\end{Shaded}

\begin{verbatim}
## [1] "site_no"            "station_nm"         "site_tp_cd"        
## [4] "dec_lat_va"         "dec_long_va"        "coord_acy_cd"      
## [7] "dec_coord_datum_cd"
\end{verbatim}

\begin{enumerate}
\def\labelenumi{\arabic{enumi}.}
\setcounter{enumi}{8}
\tightlist
\item
  What columns in the dataset contain the x and y coordinate values,
  respectively?\\
  \textgreater{} ANSWER: The x coordinate values are in the
  dec\_long\_va column. The y coordinate values are in the dec\_lat\_va
  column. \textgreater{}
\end{enumerate}

\hypertarget{convert-the-dataframe-to-a-spatial-features-sf-dataframe}{%
\subsubsection{Convert the dataframe to a spatial features (``sf'')
dataframe}\label{convert-the-dataframe-to-a-spatial-features-sf-dataframe}}

\begin{enumerate}
\def\labelenumi{\arabic{enumi}.}
\setcounter{enumi}{9}
\tightlist
\item
  Convert the dataframe to an sf dataframe.
\end{enumerate}

\begin{itemize}
\tightlist
\item
  Note: These data use the same coordinate reference system as the
  counties dataset
\end{itemize}

\begin{enumerate}
\def\labelenumi{\arabic{enumi}.}
\setcounter{enumi}{10}
\tightlist
\item
  Display the column names of the resulting sf dataframe
\end{enumerate}

\begin{Shaded}
\begin{Highlighting}[]
\CommentTok{\#10. Convert to an sf object}
\NormalTok{usgs\_gage\_sf}\OtherTok{\textless{}{-}} \FunctionTok{st\_as\_sf}\NormalTok{(usgs\_gage\_locations,}
                          \AttributeTok{coords =} \FunctionTok{c}\NormalTok{(}\StringTok{\textquotesingle{}dec\_long\_va\textquotesingle{}}\NormalTok{,}\StringTok{\textquotesingle{}dec\_lat\_va\textquotesingle{}}\NormalTok{),}
                          \AttributeTok{crs=}\DecValTok{4269}\NormalTok{)}

\CommentTok{\#11. Re{-}examine the column names}
\FunctionTok{colnames}\NormalTok{(usgs\_gage\_sf)}
\end{Highlighting}
\end{Shaded}

\begin{verbatim}
## [1] "site_no"            "station_nm"         "site_tp_cd"        
## [4] "coord_acy_cd"       "dec_coord_datum_cd" "geometry"
\end{verbatim}

\begin{enumerate}
\def\labelenumi{\arabic{enumi}.}
\setcounter{enumi}{11}
\tightlist
\item
  What new field(s) appear in the sf dataframe created? What field(s),
  if any, disappeared?
\end{enumerate}

\begin{quote}
ANSWER: The new field that appeared in the sf dataframe was the
``geometry'' column. The fields that disappeared included
``dec\_lat\_va'' and ``dec\_long\_va''.
\end{quote}

\hypertarget{plot-the-gage-locations-on-top-of-the-counties}{%
\subsubsection{Plot the gage locations on top of the
counties}\label{plot-the-gage-locations-on-top-of-the-counties}}

\begin{enumerate}
\def\labelenumi{\arabic{enumi}.}
\setcounter{enumi}{12}
\tightlist
\item
  Use \texttt{ggplot} to plot the county and gage location datasets.
\end{enumerate}

\begin{itemize}
\tightlist
\item
  Be sure the datasets are displayed in different colors
\item
  Title your plot ``NWIS Gage Locations in Nebraska''
\item
  Subtitle your plot with your name
\end{itemize}

\begin{Shaded}
\begin{Highlighting}[]
\CommentTok{\#13. Plot the gage locations atop the county features}
\NormalTok{NWIS\_Gage\_Location\_NE\_plot}\OtherTok{\textless{}{-}} \FunctionTok{ggplot}\NormalTok{()}\SpecialCharTok{+}
                              \FunctionTok{geom\_sf}\NormalTok{(}\AttributeTok{data=}\NormalTok{ NE\_counties\_sf,}\FunctionTok{aes}\NormalTok{(}\AttributeTok{color=}\StringTok{"NAME"}\NormalTok{),}
                                      \AttributeTok{alpha=}\FloatTok{0.5}\NormalTok{, }\AttributeTok{fill=}\StringTok{"white"}\NormalTok{) }\SpecialCharTok{+} 
                              \FunctionTok{scale\_color\_discrete}\NormalTok{(}\AttributeTok{name =} \ConstantTok{NULL}\NormalTok{, }
                                    \AttributeTok{labels =} \FunctionTok{c}\NormalTok{(}\StringTok{"Nebraska counties"}\NormalTok{))}\SpecialCharTok{+}
                              \FunctionTok{geom\_sf}\NormalTok{(}\AttributeTok{data=}\NormalTok{ usgs\_gage\_sf,}\AttributeTok{color=}\StringTok{"blue"}\NormalTok{,}
                                     \FunctionTok{aes}\NormalTok{(}\AttributeTok{fill=}\StringTok{"station\_nm"}\NormalTok{), }\AttributeTok{alpha=}\FloatTok{0.7}\NormalTok{)}\SpecialCharTok{+}
                              \FunctionTok{scale\_fill\_discrete}\NormalTok{(}\AttributeTok{name =} \ConstantTok{NULL}\NormalTok{, }\AttributeTok{labels =} \FunctionTok{c}\NormalTok{(}\StringTok{"USGS gage station"}\NormalTok{))}\SpecialCharTok{+}
                              \FunctionTok{labs}\NormalTok{(}\AttributeTok{title=}\StringTok{"NWIS Gage Locations in Nebraska"}\NormalTok{,}
                                   \AttributeTok{subtitle=}\StringTok{"Nancy Bao"}\NormalTok{,}
                              \AttributeTok{caption=}\StringTok{"Blue dots represent USGS gage stations. County borders are outlined in gray."}\NormalTok{)}\SpecialCharTok{+}
                              \FunctionTok{theme}\NormalTok{(}\AttributeTok{plot.title =} \FunctionTok{element\_text}\NormalTok{(}\AttributeTok{hjust =} \FloatTok{0.5}\NormalTok{),}
                              \AttributeTok{plot.subtitle =} \FunctionTok{element\_text}\NormalTok{(}\AttributeTok{hjust =} \FloatTok{0.5}\NormalTok{),}
                              \AttributeTok{plot.caption=}\FunctionTok{element\_text}\NormalTok{(}\AttributeTok{hjust=}\FloatTok{0.25}\NormalTok{))}
\FunctionTok{print}\NormalTok{(NWIS\_Gage\_Location\_NE\_plot)}
\end{Highlighting}
\end{Shaded}

\includegraphics{A09_SpatialAnalysis_files/figure-latex/Plot the spatial features-1.pdf}

\begin{Shaded}
\begin{Highlighting}[]
\CommentTok{\# I manually picked orange for the county layer and blue for the gage locations.}
\CommentTok{\#I used labs(title="",subtitle="",caption="") to label the map.}
\CommentTok{\# I renamed legend titles with scale\_color\_discrete() and scale\_fill\_discrete()}
\CommentTok{\#I added a caption to describe the colors and markers. }
\end{Highlighting}
\end{Shaded}

\hypertarget{read-in-the-gage-height-data-and-join-the-site-location-data-to-it.}{%
\subsubsection{Read in the gage height data and join the site location
data to
it.}\label{read-in-the-gage-height-data-and-join-the-site-location-data-to-it.}}

Lastly, we want to attach some gage height data to our site locations.
I've constructed a csv file listing many of the Nebraska gage sites, by
station name and site number along with stream gage heights (in meters)
recorded during the recent flood event. This file is titled
\texttt{NWIS\_SiteFlowData\_NE\_RAW.csv} and is found in the Data/Raw
folder.

\begin{enumerate}
\def\labelenumi{\arabic{enumi}.}
\setcounter{enumi}{13}
\tightlist
\item
  Read the \texttt{NWIS\_SiteFlowData\_NE\_RAW.csv} dataset in as a
  dataframe.
\item
  Show the column names .
\item
  Join our site information (already imported above) to these gage
  height data.
\end{enumerate}

\begin{itemize}
\tightlist
\item
  The \texttt{site\_no} and \texttt{station\_nm} can both/either serve
  as joining attributes.
\item
  Construct this join so that the result only includes spatial features
  where both tables have data.
\end{itemize}

\begin{enumerate}
\def\labelenumi{\arabic{enumi}.}
\setcounter{enumi}{16}
\tightlist
\item
  Show the column names in this resulting spatial features object
\item
  Show the dimensions of the resulting joined dataframe
\end{enumerate}

\begin{Shaded}
\begin{Highlighting}[]
\CommentTok{\#14. Read the site flow data into a data frame}
\NormalTok{nwis\_site\_flow }\OtherTok{\textless{}{-}}\FunctionTok{read.csv}\NormalTok{(}\StringTok{"./Data/Raw/NWIS\_SiteFlowData\_NE\_RAW.csv"}\NormalTok{)}

\CommentTok{\#15. Show the column names}
\FunctionTok{colnames}\NormalTok{(nwis\_site\_flow)}
\end{Highlighting}
\end{Shaded}

\begin{verbatim}
## [1] "site_no"    "station_nm" "date"       "gage_ht"
\end{verbatim}

\begin{Shaded}
\begin{Highlighting}[]
\CommentTok{\#16. Join location data to it}
\NormalTok{site\_flow\_location\_join}\OtherTok{\textless{}{-}}\NormalTok{usgs\_gage\_sf }\SpecialCharTok{\%\textgreater{}\%}
              \FunctionTok{inner\_join}\NormalTok{(nwis\_site\_flow, }\AttributeTok{by=}\FunctionTok{c}\NormalTok{(}\StringTok{\textquotesingle{}station\_nm\textquotesingle{}}\NormalTok{,}\StringTok{\textquotesingle{}site\_no\textquotesingle{}}\NormalTok{))}
\CommentTok{\#I did an inner\_join, so results only had spatial features where both tables have data}

\CommentTok{\#17. Show the column names of the joined dataset}
\FunctionTok{colnames}\NormalTok{(site\_flow\_location\_join)}
\end{Highlighting}
\end{Shaded}

\begin{verbatim}
## [1] "site_no"            "station_nm"         "site_tp_cd"        
## [4] "coord_acy_cd"       "dec_coord_datum_cd" "date"              
## [7] "gage_ht"            "geometry"
\end{verbatim}

\begin{Shaded}
\begin{Highlighting}[]
\CommentTok{\#18. Show the dimensions of this joined dataset}
\FunctionTok{dim}\NormalTok{(site\_flow\_location\_join)}
\end{Highlighting}
\end{Shaded}

\begin{verbatim}
## [1] 136   8
\end{verbatim}

\hypertarget{map-the-pattern-of-gage-height-data}{%
\subsubsection{Map the pattern of gage height
data}\label{map-the-pattern-of-gage-height-data}}

Now we can examine where the flooding appears most acute by visualizing
gage heights spatially. 19. Plot the gage sites on top of counties
(using \texttt{mapview}, \texttt{ggplot}, or \texttt{leaflet}) * Show
the magnitude of gage height by color, shape, other visualization
technique.

\begin{Shaded}
\begin{Highlighting}[]
\CommentTok{\#Map the points, sized by gage height}
\NormalTok{gage\_heights\_map}\OtherTok{\textless{}{-}}\FunctionTok{ggplot}\NormalTok{()}\SpecialCharTok{+}
                    \FunctionTok{geom\_sf}\NormalTok{(}\AttributeTok{data=}\NormalTok{NE\_counties\_sf,}\AttributeTok{color=}\StringTok{"black"}\NormalTok{,}
                          \AttributeTok{fill=}\StringTok{"white"}\NormalTok{)}\SpecialCharTok{+}
                    \FunctionTok{geom\_sf}\NormalTok{(}\AttributeTok{data=}\NormalTok{site\_flow\_location\_join,}
                                \FunctionTok{aes}\NormalTok{(}\AttributeTok{color=}\NormalTok{ gage\_ht),}\AttributeTok{size=}\FloatTok{1.5}\NormalTok{,}\AttributeTok{alpha=}\DecValTok{1}\NormalTok{)}\SpecialCharTok{+}
                    \FunctionTok{scale\_color\_continuous}\NormalTok{(}\AttributeTok{name=}\StringTok{"USGS gage height (in meters)"}\NormalTok{)}\SpecialCharTok{+}
                    \FunctionTok{labs}\NormalTok{(}\AttributeTok{title=}\StringTok{"NWIS Gage Locations in Nebraska"}\NormalTok{,}
                                   \AttributeTok{subtitle=}\StringTok{"Nancy Bao"}\NormalTok{)}\SpecialCharTok{+}
                    \FunctionTok{theme}\NormalTok{(}\AttributeTok{plot.title =} \FunctionTok{element\_text}\NormalTok{(}\AttributeTok{hjust =} \FloatTok{0.5}\NormalTok{),}
                              \AttributeTok{plot.subtitle =} \FunctionTok{element\_text}\NormalTok{(}\AttributeTok{hjust =} \FloatTok{0.5}\NormalTok{))}

\FunctionTok{print}\NormalTok{(gage\_heights\_map)}
\end{Highlighting}
\end{Shaded}

\includegraphics{A09_SpatialAnalysis_files/figure-latex/unnamed-chunk-2-1.pdf}

\begin{Shaded}
\begin{Highlighting}[]
\CommentTok{\#I distinguished gage height by a blue gradient:}
\CommentTok{\#with the lowest heights the darkest and the highest heights the lightest blues.}
\CommentTok{\#I used hjust=0.5 to center the title}
\CommentTok{\#I renamed the legend with the scale\_color\_continuous()}
\end{Highlighting}
\end{Shaded}

\begin{center}\rule{0.5\linewidth}{0.5pt}\end{center}

\hypertarget{spatial-analysis}{%
\subsection{SPATIAL ANALYSIS}\label{spatial-analysis}}

Up next we will do some spatial analysis with our data. To prepare for
this, we should transform our data into a projected coordinate system.
We'll choose UTM Zone 14N (EPGS = 32614).

\hypertarget{transform-the-counties-and-gage-site-datasets-to-utm-zone-14n}{%
\subsubsection{Transform the counties and gage site datasets to UTM Zone
14N}\label{transform-the-counties-and-gage-site-datasets-to-utm-zone-14n}}

\begin{enumerate}
\def\labelenumi{\arabic{enumi}.}
\setcounter{enumi}{19}
\tightlist
\item
  Transform the counties and gage sf datasets to UTM Zone 14N (EPGS =
  32614).
\item
  Using \texttt{mapview} or \texttt{ggplot}, plot the data so that each
  can be seen as different colors
\end{enumerate}

\begin{Shaded}
\begin{Highlighting}[]
\CommentTok{\#20 Transform the counties and gage location datasets to UTM Zone 14}
\NormalTok{NE\_counties\_sf\_utm }\OtherTok{\textless{}{-}}\FunctionTok{st\_transform}\NormalTok{(NE\_counties\_sf, }\AttributeTok{crs =}\DecValTok{32614}\NormalTok{)}
\NormalTok{nwis\_gage\_location\_sf\_utm }\OtherTok{\textless{}{-}}\FunctionTok{st\_transform}\NormalTok{(usgs\_gage\_sf , }\AttributeTok{crs =} \DecValTok{32614}\NormalTok{)}

\CommentTok{\#21 Plot the data}
\CommentTok{\#I used ggplot to plot the transformed datasets and used scale\_fill\_discrete to rename legend}
\NormalTok{gage\_county\_map}\OtherTok{\textless{}{-}}\FunctionTok{ggplot}\NormalTok{()}\SpecialCharTok{+}
                  \FunctionTok{geom\_sf}\NormalTok{(}\AttributeTok{data=}\NormalTok{NE\_counties\_sf\_utm,}
                         \AttributeTok{color=}\StringTok{"black"}\NormalTok{,}\AttributeTok{fill=}\StringTok{"white"}\NormalTok{)}\SpecialCharTok{+}
                 \FunctionTok{geom\_sf}\NormalTok{(}\AttributeTok{data=}\NormalTok{nwis\_gage\_location\_sf\_utm,}
                         \AttributeTok{color=}\StringTok{"green"}\NormalTok{,}\AttributeTok{size=}\FloatTok{1.5}\NormalTok{,}\FunctionTok{aes}\NormalTok{(}\AttributeTok{fill=}\StringTok{"USGS gage station"}\NormalTok{))}\SpecialCharTok{+}
                 \FunctionTok{scale\_fill\_discrete}\NormalTok{(}\AttributeTok{name=}\StringTok{""}\NormalTok{)}\SpecialCharTok{+}
                 \FunctionTok{labs}\NormalTok{(}\AttributeTok{title=}\StringTok{"NWIS Gage Locations in Nebraska"}\NormalTok{,}
                                   \AttributeTok{subtitle=}\StringTok{"Nancy Bao"}\NormalTok{,}
                      \AttributeTok{caption=}\StringTok{"NE counties are outlined in black."}\NormalTok{)}\SpecialCharTok{+}
                \FunctionTok{theme}\NormalTok{(}\AttributeTok{plot.title =} \FunctionTok{element\_text}\NormalTok{(}\AttributeTok{hjust =} \FloatTok{0.5}\NormalTok{),}
                              \AttributeTok{plot.subtitle =} \FunctionTok{element\_text}\NormalTok{(}\AttributeTok{hjust =} \FloatTok{0.5}\NormalTok{))}
\FunctionTok{print}\NormalTok{(gage\_county\_map)}
\end{Highlighting}
\end{Shaded}

\includegraphics{A09_SpatialAnalysis_files/figure-latex/unnamed-chunk-3-1.pdf}

\hypertarget{select-the-gages-falling-within-a-given-county}{%
\subsubsection{Select the gages falling within a given
county}\label{select-the-gages-falling-within-a-given-county}}

Now let's zoom into a particular county and examine the gages located
there. 22. Select Lancaster county from your county sf dataframe 23.
Select the gage sites falling \texttt{within} that county * Use either
matrix subsetting or tidy filtering 24. Create a plot showing: * all
Nebraska counties, * the selected county, * and the gage sites in that
county

\begin{Shaded}
\begin{Highlighting}[]
\CommentTok{\#22 Select the county}
\CommentTok{\#I used the UTM transformed datasets }
\NormalTok{Lancaster\_co\_sf\_utm}\OtherTok{\textless{}{-}}\NormalTok{ NE\_counties\_sf\_utm }\SpecialCharTok{\%\textgreater{}\%}
                    \FunctionTok{filter}\NormalTok{(NAME}\SpecialCharTok{==}\StringTok{"Lancaster"}\NormalTok{)}

\CommentTok{\#23 Select gages within the selected county}
\NormalTok{selected\_gages\_sf\_utm}\OtherTok{\textless{}{-}}\NormalTok{ nwis\_gage\_location\_sf\_utm[Lancaster\_co\_sf\_utm,]}
\CommentTok{\#I used the matrix subsetting to get the gage locations within Lancaster Co. }

\CommentTok{\#24 Plot}
\CommentTok{\#I used the UTM transformed datasets to plot}
\NormalTok{Lancaster\_plot }\OtherTok{\textless{}{-}} \FunctionTok{ggplot}\NormalTok{()}\SpecialCharTok{+}
                  \FunctionTok{geom\_sf}\NormalTok{(}\AttributeTok{data=}\NormalTok{NE\_counties\_sf\_utm,}
                          \AttributeTok{fill=}\StringTok{"light blue"}\NormalTok{,}\AttributeTok{alpha=}\FloatTok{0.5}\NormalTok{)}\SpecialCharTok{+}
                  \FunctionTok{geom\_sf}\NormalTok{(}\AttributeTok{data=}\NormalTok{Lancaster\_co\_sf\_utm,}
                          \AttributeTok{fill=}\StringTok{"orange"}\NormalTok{,}\FunctionTok{aes}\NormalTok{(}\AttributeTok{color=}\StringTok{"NAME"}\NormalTok{))}\SpecialCharTok{+}
                  \FunctionTok{scale\_color\_discrete}\NormalTok{(}\AttributeTok{name =} \ConstantTok{NULL}\NormalTok{, }
                                       \AttributeTok{labels =} \FunctionTok{c}\NormalTok{(}\StringTok{"Lancaster County"}\NormalTok{))}\SpecialCharTok{+}
                  \FunctionTok{geom\_sf}\NormalTok{(}\AttributeTok{data=}\NormalTok{selected\_gages\_sf\_utm,}
                          \FunctionTok{aes}\NormalTok{(}\AttributeTok{fill=}\StringTok{\textquotesingle{}station\_nm\textquotesingle{}}\NormalTok{),}\AttributeTok{size=}\FloatTok{0.5}\NormalTok{, }\AttributeTok{alpha=}\FloatTok{0.7}\NormalTok{)}\SpecialCharTok{+}
                  \FunctionTok{scale\_fill\_discrete}\NormalTok{(}\AttributeTok{name =} \ConstantTok{NULL}\NormalTok{, }\AttributeTok{labels =} \FunctionTok{c}\NormalTok{(}\StringTok{"USGS gage station"}\NormalTok{))}\SpecialCharTok{+}
                 \FunctionTok{labs}\NormalTok{(}\AttributeTok{title=}\StringTok{"NWIS Gage Locations in Nebraska"}\NormalTok{,}
                                   \AttributeTok{subtitle=}\StringTok{"Nancy Bao"}\NormalTok{,}
                      \AttributeTok{caption=}\StringTok{"All county borders in Nebraska are outlined in gray and filled in light blue except Lancaster County, which is outlined in pink and filled in orange."}\NormalTok{)}\SpecialCharTok{+}
                              \FunctionTok{theme}\NormalTok{(}\AttributeTok{plot.title =} \FunctionTok{element\_text}\NormalTok{(}\AttributeTok{hjust =} \FloatTok{0.5}\NormalTok{),}
                              \AttributeTok{plot.subtitle =} \FunctionTok{element\_text}\NormalTok{(}\AttributeTok{hjust =} \FloatTok{0.5}\NormalTok{),}
                              \AttributeTok{plot.caption=}\FunctionTok{element\_text}\NormalTok{(}\AttributeTok{hjust=}\FloatTok{0.25}\NormalTok{))}
\FunctionTok{print}\NormalTok{(Lancaster\_plot)}
\end{Highlighting}
\end{Shaded}

\includegraphics{A09_SpatialAnalysis_files/figure-latex/unnamed-chunk-4-1.pdf}

\begin{Shaded}
\begin{Highlighting}[]
\CommentTok{\#I layered each sf dataframe to show all the NE counties, }
\CommentTok{\# distinguish Lancaster Co., and show the gages within Lancaster Co. }

\CommentTok{\#I used scale\_color\_discrete and scale\_fill\_discrete to rename the legends }
\CommentTok{\#to show Lancaster Co. and the gage stations in that county. }

\CommentTok{\#I adjusted the labels so that they would be centered and}
\CommentTok{\#I added a caption to the bottom of the graph to explain what the blue blocks mean. }
\end{Highlighting}
\end{Shaded}


\end{document}
