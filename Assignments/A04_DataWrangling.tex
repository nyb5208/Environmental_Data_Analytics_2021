% Options for packages loaded elsewhere
\PassOptionsToPackage{unicode}{hyperref}
\PassOptionsToPackage{hyphens}{url}
%
\documentclass[
]{article}
\usepackage{lmodern}
\usepackage{amsmath}
\usepackage{ifxetex,ifluatex}
\ifnum 0\ifxetex 1\fi\ifluatex 1\fi=0 % if pdftex
  \usepackage[T1]{fontenc}
  \usepackage[utf8]{inputenc}
  \usepackage{textcomp} % provide euro and other symbols
  \usepackage{amssymb}
\else % if luatex or xetex
  \usepackage{unicode-math}
  \defaultfontfeatures{Scale=MatchLowercase}
  \defaultfontfeatures[\rmfamily]{Ligatures=TeX,Scale=1}
\fi
% Use upquote if available, for straight quotes in verbatim environments
\IfFileExists{upquote.sty}{\usepackage{upquote}}{}
\IfFileExists{microtype.sty}{% use microtype if available
  \usepackage[]{microtype}
  \UseMicrotypeSet[protrusion]{basicmath} % disable protrusion for tt fonts
}{}
\makeatletter
\@ifundefined{KOMAClassName}{% if non-KOMA class
  \IfFileExists{parskip.sty}{%
    \usepackage{parskip}
  }{% else
    \setlength{\parindent}{0pt}
    \setlength{\parskip}{6pt plus 2pt minus 1pt}}
}{% if KOMA class
  \KOMAoptions{parskip=half}}
\makeatother
\usepackage{xcolor}
\IfFileExists{xurl.sty}{\usepackage{xurl}}{} % add URL line breaks if available
\IfFileExists{bookmark.sty}{\usepackage{bookmark}}{\usepackage{hyperref}}
\hypersetup{
  pdftitle={Assignment 4: Data Wrangling},
  pdfauthor={Nancy Bao},
  hidelinks,
  pdfcreator={LaTeX via pandoc}}
\urlstyle{same} % disable monospaced font for URLs
\usepackage[margin=2.54cm]{geometry}
\usepackage{color}
\usepackage{fancyvrb}
\newcommand{\VerbBar}{|}
\newcommand{\VERB}{\Verb[commandchars=\\\{\}]}
\DefineVerbatimEnvironment{Highlighting}{Verbatim}{commandchars=\\\{\}}
% Add ',fontsize=\small' for more characters per line
\usepackage{framed}
\definecolor{shadecolor}{RGB}{248,248,248}
\newenvironment{Shaded}{\begin{snugshade}}{\end{snugshade}}
\newcommand{\AlertTok}[1]{\textcolor[rgb]{0.94,0.16,0.16}{#1}}
\newcommand{\AnnotationTok}[1]{\textcolor[rgb]{0.56,0.35,0.01}{\textbf{\textit{#1}}}}
\newcommand{\AttributeTok}[1]{\textcolor[rgb]{0.77,0.63,0.00}{#1}}
\newcommand{\BaseNTok}[1]{\textcolor[rgb]{0.00,0.00,0.81}{#1}}
\newcommand{\BuiltInTok}[1]{#1}
\newcommand{\CharTok}[1]{\textcolor[rgb]{0.31,0.60,0.02}{#1}}
\newcommand{\CommentTok}[1]{\textcolor[rgb]{0.56,0.35,0.01}{\textit{#1}}}
\newcommand{\CommentVarTok}[1]{\textcolor[rgb]{0.56,0.35,0.01}{\textbf{\textit{#1}}}}
\newcommand{\ConstantTok}[1]{\textcolor[rgb]{0.00,0.00,0.00}{#1}}
\newcommand{\ControlFlowTok}[1]{\textcolor[rgb]{0.13,0.29,0.53}{\textbf{#1}}}
\newcommand{\DataTypeTok}[1]{\textcolor[rgb]{0.13,0.29,0.53}{#1}}
\newcommand{\DecValTok}[1]{\textcolor[rgb]{0.00,0.00,0.81}{#1}}
\newcommand{\DocumentationTok}[1]{\textcolor[rgb]{0.56,0.35,0.01}{\textbf{\textit{#1}}}}
\newcommand{\ErrorTok}[1]{\textcolor[rgb]{0.64,0.00,0.00}{\textbf{#1}}}
\newcommand{\ExtensionTok}[1]{#1}
\newcommand{\FloatTok}[1]{\textcolor[rgb]{0.00,0.00,0.81}{#1}}
\newcommand{\FunctionTok}[1]{\textcolor[rgb]{0.00,0.00,0.00}{#1}}
\newcommand{\ImportTok}[1]{#1}
\newcommand{\InformationTok}[1]{\textcolor[rgb]{0.56,0.35,0.01}{\textbf{\textit{#1}}}}
\newcommand{\KeywordTok}[1]{\textcolor[rgb]{0.13,0.29,0.53}{\textbf{#1}}}
\newcommand{\NormalTok}[1]{#1}
\newcommand{\OperatorTok}[1]{\textcolor[rgb]{0.81,0.36,0.00}{\textbf{#1}}}
\newcommand{\OtherTok}[1]{\textcolor[rgb]{0.56,0.35,0.01}{#1}}
\newcommand{\PreprocessorTok}[1]{\textcolor[rgb]{0.56,0.35,0.01}{\textit{#1}}}
\newcommand{\RegionMarkerTok}[1]{#1}
\newcommand{\SpecialCharTok}[1]{\textcolor[rgb]{0.00,0.00,0.00}{#1}}
\newcommand{\SpecialStringTok}[1]{\textcolor[rgb]{0.31,0.60,0.02}{#1}}
\newcommand{\StringTok}[1]{\textcolor[rgb]{0.31,0.60,0.02}{#1}}
\newcommand{\VariableTok}[1]{\textcolor[rgb]{0.00,0.00,0.00}{#1}}
\newcommand{\VerbatimStringTok}[1]{\textcolor[rgb]{0.31,0.60,0.02}{#1}}
\newcommand{\WarningTok}[1]{\textcolor[rgb]{0.56,0.35,0.01}{\textbf{\textit{#1}}}}
\usepackage{graphicx}
\makeatletter
\def\maxwidth{\ifdim\Gin@nat@width>\linewidth\linewidth\else\Gin@nat@width\fi}
\def\maxheight{\ifdim\Gin@nat@height>\textheight\textheight\else\Gin@nat@height\fi}
\makeatother
% Scale images if necessary, so that they will not overflow the page
% margins by default, and it is still possible to overwrite the defaults
% using explicit options in \includegraphics[width, height, ...]{}
\setkeys{Gin}{width=\maxwidth,height=\maxheight,keepaspectratio}
% Set default figure placement to htbp
\makeatletter
\def\fps@figure{htbp}
\makeatother
\setlength{\emergencystretch}{3em} % prevent overfull lines
\providecommand{\tightlist}{%
  \setlength{\itemsep}{0pt}\setlength{\parskip}{0pt}}
\setcounter{secnumdepth}{-\maxdimen} % remove section numbering
\ifluatex
  \usepackage{selnolig}  % disable illegal ligatures
\fi

\title{Assignment 4: Data Wrangling}
\author{Nancy Bao}
\date{}

\begin{document}
\maketitle

\hypertarget{overview}{%
\subsection{OVERVIEW}\label{overview}}

This exercise accompanies the lessons in Environmental Data Analytics on
Data Wrangling

\hypertarget{directions}{%
\subsection{Directions}\label{directions}}

\begin{enumerate}
\def\labelenumi{\arabic{enumi}.}
\tightlist
\item
  Change ``Student Name'' on line 3 (above) with your name.
\item
  Work through the steps, \textbf{creating code and output} that fulfill
  each instruction.
\item
  Be sure to \textbf{answer the questions} in this assignment document.
\item
  When you have completed the assignment, \textbf{Knit} the text and
  code into a single PDF file.
\item
  After Knitting, submit the completed exercise (PDF file) to the
  dropbox in Sakai. Add your last name into the file name (e.g.,
  ``Fay\_A04\_DataWrangling.Rmd'') prior to submission.
\end{enumerate}

The completed exercise is due on Tuesday, Feb 16 @ 11:59pm.

\hypertarget{set-up-your-session}{%
\subsection{Set up your session}\label{set-up-your-session}}

\begin{enumerate}
\def\labelenumi{\arabic{enumi}.}
\item
  Check your working directory, load the \texttt{tidyverse} and
  \texttt{lubridate} packages, and upload all four raw data files
  associated with the EPA Air dataset. See the README file for the EPA
  air datasets for more information (especially if you have not worked
  with air quality data previously).
\item
  Explore the dimensions, column names, and structure of the datasets.
\end{enumerate}

\begin{Shaded}
\begin{Highlighting}[]
\CommentTok{\#1}
\CommentTok{\#Check the working directory}
\FunctionTok{getwd}\NormalTok{()}
\end{Highlighting}
\end{Shaded}

\begin{verbatim}
## [1] "/Users/Nancy/Desktop/Semester 4/ENV 872L/Environmental_Data_Analytics_2021"
\end{verbatim}

\begin{Shaded}
\begin{Highlighting}[]
\CommentTok{\#working directory is already set to the main folder: }
\CommentTok{\#"/Users/Nancy/Desktop/Semester 4/ENV 872L/Environmental\_Data\_Analytics\_2021"}

\CommentTok{\#Load the packages}
\FunctionTok{library}\NormalTok{(tidyverse)}
\FunctionTok{library}\NormalTok{(lubridate)}

\CommentTok{\#Import the four EPA air datasets; set relative file path}
\NormalTok{epa\_air\_2018}\OtherTok{\textless{}{-}}\FunctionTok{read.csv}\NormalTok{(}\StringTok{"Data/Raw/EPAair\_O3\_NC2018\_raw.csv"}\NormalTok{, }
                       \AttributeTok{stringsAsFactors =} \ConstantTok{TRUE}\NormalTok{)}
\NormalTok{epa\_air\_2019}\OtherTok{\textless{}{-}}\FunctionTok{read.csv}\NormalTok{(}\StringTok{"Data/Raw/EPAair\_O3\_NC2019\_raw.csv"}\NormalTok{, }
                       \AttributeTok{stringsAsFactors =} \ConstantTok{TRUE}\NormalTok{)}
\NormalTok{epa\_air\_PM2}\FloatTok{.5}\NormalTok{\_2018}\OtherTok{\textless{}{-}}\FunctionTok{read.csv}\NormalTok{(}\StringTok{"Data/Raw/EPAair\_PM25\_NC2018\_raw.csv"}\NormalTok{, }
                             \AttributeTok{stringsAsFactors =} \ConstantTok{TRUE}\NormalTok{)}
\NormalTok{epa\_air\_PM2}\FloatTok{.5}\NormalTok{\_2019}\OtherTok{\textless{}{-}}\FunctionTok{read.csv}\NormalTok{(}\StringTok{"Data/Raw/EPAair\_PM25\_NC2019\_raw.csv"}\NormalTok{, }
                             \AttributeTok{stringsAsFactors =} \ConstantTok{TRUE}\NormalTok{)}
\CommentTok{\#2 }
\DocumentationTok{\#\# Explore dimensions}
\FunctionTok{dim}\NormalTok{(epa\_air\_2018) }\CommentTok{\# 9737 rows and 20 columns}
\end{Highlighting}
\end{Shaded}

\begin{verbatim}
## [1] 9737   20
\end{verbatim}

\begin{Shaded}
\begin{Highlighting}[]
\FunctionTok{dim}\NormalTok{(epa\_air\_2019) }\CommentTok{\# 10592 rows and 20 columns}
\end{Highlighting}
\end{Shaded}

\begin{verbatim}
## [1] 10592    20
\end{verbatim}

\begin{Shaded}
\begin{Highlighting}[]
\FunctionTok{dim}\NormalTok{(epa\_air\_PM2}\FloatTok{.5}\NormalTok{\_2018) }\CommentTok{\# 8983 rows and 20 columns}
\end{Highlighting}
\end{Shaded}

\begin{verbatim}
## [1] 8983   20
\end{verbatim}

\begin{Shaded}
\begin{Highlighting}[]
\FunctionTok{dim}\NormalTok{(epa\_air\_PM2}\FloatTok{.5}\NormalTok{\_2019) }\CommentTok{\#8581 rows and 20 columns}
\end{Highlighting}
\end{Shaded}

\begin{verbatim}
## [1] 8581   20
\end{verbatim}

\begin{Shaded}
\begin{Highlighting}[]
\DocumentationTok{\#\# Explore column names}
\FunctionTok{colnames}\NormalTok{(epa\_air\_2018)}
\end{Highlighting}
\end{Shaded}

\begin{verbatim}
##  [1] "Date"                                
##  [2] "Source"                              
##  [3] "Site.ID"                             
##  [4] "POC"                                 
##  [5] "Daily.Max.8.hour.Ozone.Concentration"
##  [6] "UNITS"                               
##  [7] "DAILY_AQI_VALUE"                     
##  [8] "Site.Name"                           
##  [9] "DAILY_OBS_COUNT"                     
## [10] "PERCENT_COMPLETE"                    
## [11] "AQS_PARAMETER_CODE"                  
## [12] "AQS_PARAMETER_DESC"                  
## [13] "CBSA_CODE"                           
## [14] "CBSA_NAME"                           
## [15] "STATE_CODE"                          
## [16] "STATE"                               
## [17] "COUNTY_CODE"                         
## [18] "COUNTY"                              
## [19] "SITE_LATITUDE"                       
## [20] "SITE_LONGITUDE"
\end{verbatim}

\begin{Shaded}
\begin{Highlighting}[]
\FunctionTok{colnames}\NormalTok{(epa\_air\_2019)}
\end{Highlighting}
\end{Shaded}

\begin{verbatim}
##  [1] "Date"                                
##  [2] "Source"                              
##  [3] "Site.ID"                             
##  [4] "POC"                                 
##  [5] "Daily.Max.8.hour.Ozone.Concentration"
##  [6] "UNITS"                               
##  [7] "DAILY_AQI_VALUE"                     
##  [8] "Site.Name"                           
##  [9] "DAILY_OBS_COUNT"                     
## [10] "PERCENT_COMPLETE"                    
## [11] "AQS_PARAMETER_CODE"                  
## [12] "AQS_PARAMETER_DESC"                  
## [13] "CBSA_CODE"                           
## [14] "CBSA_NAME"                           
## [15] "STATE_CODE"                          
## [16] "STATE"                               
## [17] "COUNTY_CODE"                         
## [18] "COUNTY"                              
## [19] "SITE_LATITUDE"                       
## [20] "SITE_LONGITUDE"
\end{verbatim}

\begin{Shaded}
\begin{Highlighting}[]
\FunctionTok{colnames}\NormalTok{(epa\_air\_PM2}\FloatTok{.5}\NormalTok{\_2018)}
\end{Highlighting}
\end{Shaded}

\begin{verbatim}
##  [1] "Date"                           "Source"                        
##  [3] "Site.ID"                        "POC"                           
##  [5] "Daily.Mean.PM2.5.Concentration" "UNITS"                         
##  [7] "DAILY_AQI_VALUE"                "Site.Name"                     
##  [9] "DAILY_OBS_COUNT"                "PERCENT_COMPLETE"              
## [11] "AQS_PARAMETER_CODE"             "AQS_PARAMETER_DESC"            
## [13] "CBSA_CODE"                      "CBSA_NAME"                     
## [15] "STATE_CODE"                     "STATE"                         
## [17] "COUNTY_CODE"                    "COUNTY"                        
## [19] "SITE_LATITUDE"                  "SITE_LONGITUDE"
\end{verbatim}

\begin{Shaded}
\begin{Highlighting}[]
\FunctionTok{colnames}\NormalTok{(epa\_air\_PM2}\FloatTok{.5}\NormalTok{\_2019)}
\end{Highlighting}
\end{Shaded}

\begin{verbatim}
##  [1] "Date"                           "Source"                        
##  [3] "Site.ID"                        "POC"                           
##  [5] "Daily.Mean.PM2.5.Concentration" "UNITS"                         
##  [7] "DAILY_AQI_VALUE"                "Site.Name"                     
##  [9] "DAILY_OBS_COUNT"                "PERCENT_COMPLETE"              
## [11] "AQS_PARAMETER_CODE"             "AQS_PARAMETER_DESC"            
## [13] "CBSA_CODE"                      "CBSA_NAME"                     
## [15] "STATE_CODE"                     "STATE"                         
## [17] "COUNTY_CODE"                    "COUNTY"                        
## [19] "SITE_LATITUDE"                  "SITE_LONGITUDE"
\end{verbatim}

\begin{Shaded}
\begin{Highlighting}[]
\CommentTok{\#The PM2.5 datasets for 2018 and 2019 has  a column for Daily.Mean.PM2.5.Concentration }
\CommentTok{\#where the remaining datasets instead, have a "Daily.Max.8.hour.Ozone.Concentration"column.}


\DocumentationTok{\#\# Explore structures }

\CommentTok{\#https://www.rdocumentation.org/packages/utils/versions/3.6.2/topics/str}
\CommentTok{\# I used the URL above to read more on the strict.width and width option }
\CommentTok{\# to figure out how to make the width of str() output fit page}
\CommentTok{\# I wanted the output to fit 8.5x11 page, so I chose width 85}

\CommentTok{\#}
\FunctionTok{str}\NormalTok{(epa\_air\_2018,}\AttributeTok{strict.width=}\StringTok{"cut"}\NormalTok{,}\AttributeTok{width=}\DecValTok{85}\NormalTok{)}
\end{Highlighting}
\end{Shaded}

\begin{verbatim}
## 'data.frame':    9737 obs. of  20 variables:
##  $ Date                                : Factor w/ 364 levels "01/01/2018","01/02/"..
##  $ Source                              : Factor w/ 1 level "AQS": 1 1 1 1 1 1 1 1 1..
##  $ Site.ID                             : int  370030005 370030005 370030005 3700300..
##  $ POC                                 : int  1 1 1 1 1 1 1 1 1 1 ...
##  $ Daily.Max.8.hour.Ozone.Concentration: num  0.043 0.046 0.047 0.049 0.047 0.03 0...
##  $ UNITS                               : Factor w/ 1 level "ppm": 1 1 1 1 1 1 1 1 1..
##  $ DAILY_AQI_VALUE                     : int  40 43 44 45 44 28 33 41 45 40 ...
##  $ Site.Name                           : Factor w/ 40 levels "","Beaufort",..: 35 3..
##  $ DAILY_OBS_COUNT                     : int  17 17 17 17 17 17 17 17 17 17 ...
##  $ PERCENT_COMPLETE                    : num  100 100 100 100 100 100 100 100 100 1..
##  $ AQS_PARAMETER_CODE                  : int  44201 44201 44201 44201 44201 44201 4..
##  $ AQS_PARAMETER_DESC                  : Factor w/ 1 level "Ozone": 1 1 1 1 1 1 1 1..
##  $ CBSA_CODE                           : int  25860 25860 25860 25860 25860 25860 2..
##  $ CBSA_NAME                           : Factor w/ 17 levels "","Asheville, NC",..:..
##  $ STATE_CODE                          : int  37 37 37 37 37 37 37 37 37 37 ...
##  $ STATE                               : Factor w/ 1 level "North Carolina": 1 1 1 ..
##  $ COUNTY_CODE                         : int  3 3 3 3 3 3 3 3 3 3 ...
##  $ COUNTY                              : Factor w/ 32 levels "Alexander","Avery",....
##  $ SITE_LATITUDE                       : num  35.9 35.9 35.9 35.9 35.9 ...
##  $ SITE_LONGITUDE                      : num  -81.2 -81.2 -81.2 -81.2 -81.2 ...
\end{verbatim}

\begin{Shaded}
\begin{Highlighting}[]
\FunctionTok{str}\NormalTok{(epa\_air\_2019,  }\AttributeTok{strict.width=}\StringTok{"cut"}\NormalTok{,}\AttributeTok{width=}\DecValTok{85}\NormalTok{)}
\end{Highlighting}
\end{Shaded}

\begin{verbatim}
## 'data.frame':    10592 obs. of  20 variables:
##  $ Date                                : Factor w/ 365 levels "01/01/2019","01/02/"..
##  $ Source                              : Factor w/ 2 levels "AirNow","AQS": 1 1 1 1..
##  $ Site.ID                             : int  370030005 370030005 370030005 3700300..
##  $ POC                                 : int  1 1 1 1 1 1 1 1 1 1 ...
##  $ Daily.Max.8.hour.Ozone.Concentration: num  0.029 0.018 0.016 0.022 0.037 0.037 0..
##  $ UNITS                               : Factor w/ 1 level "ppm": 1 1 1 1 1 1 1 1 1..
##  $ DAILY_AQI_VALUE                     : int  27 17 15 20 34 34 27 35 35 28 ...
##  $ Site.Name                           : Factor w/ 38 levels "","Beaufort",..: 33 3..
##  $ DAILY_OBS_COUNT                     : int  24 24 24 24 24 24 24 24 24 24 ...
##  $ PERCENT_COMPLETE                    : num  100 100 100 100 100 100 100 100 100 1..
##  $ AQS_PARAMETER_CODE                  : int  44201 44201 44201 44201 44201 44201 4..
##  $ AQS_PARAMETER_DESC                  : Factor w/ 1 level "Ozone": 1 1 1 1 1 1 1 1..
##  $ CBSA_CODE                           : int  25860 25860 25860 25860 25860 25860 2..
##  $ CBSA_NAME                           : Factor w/ 15 levels "","Asheville, NC",..:..
##  $ STATE_CODE                          : int  37 37 37 37 37 37 37 37 37 37 ...
##  $ STATE                               : Factor w/ 1 level "North Carolina": 1 1 1 ..
##  $ COUNTY_CODE                         : int  3 3 3 3 3 3 3 3 3 3 ...
##  $ COUNTY                              : Factor w/ 30 levels "Alexander","Avery",....
##  $ SITE_LATITUDE                       : num  35.9 35.9 35.9 35.9 35.9 ...
##  $ SITE_LONGITUDE                      : num  -81.2 -81.2 -81.2 -81.2 -81.2 ...
\end{verbatim}

\begin{Shaded}
\begin{Highlighting}[]
\FunctionTok{str}\NormalTok{(epa\_air\_PM2}\FloatTok{.5}\NormalTok{\_2018, }\AttributeTok{strict.width=}\StringTok{"cut"}\NormalTok{,}\AttributeTok{width=}\DecValTok{85}\NormalTok{)}
\end{Highlighting}
\end{Shaded}

\begin{verbatim}
## 'data.frame':    8983 obs. of  20 variables:
##  $ Date                          : Factor w/ 365 levels "01/01/2018","01/02/2018",...
##  $ Source                        : Factor w/ 1 level "AQS": 1 1 1 1 1 1 1 1 1 1 ...
##  $ Site.ID                       : int  370110002 370110002 370110002 370110002 370..
##  $ POC                           : int  1 1 1 1 1 1 1 1 1 1 ...
##  $ Daily.Mean.PM2.5.Concentration: num  2.9 3.7 5.3 0.8 2.5 4.5 1.8 2.5 4.2 1.7 ...
##  $ UNITS                         : Factor w/ 1 level "ug/m3 LC": 1 1 1 1 1 1 1 1 1 ..
##  $ DAILY_AQI_VALUE               : int  12 15 22 3 10 19 8 10 18 7 ...
##  $ Site.Name                     : Factor w/ 25 levels "","Blackstone",..: 15 15 15..
##  $ DAILY_OBS_COUNT               : int  1 1 1 1 1 1 1 1 1 1 ...
##  $ PERCENT_COMPLETE              : num  100 100 100 100 100 100 100 100 100 100 ...
##  $ AQS_PARAMETER_CODE            : int  88502 88502 88502 88502 88502 88502 88502 8..
##  $ AQS_PARAMETER_DESC            : Factor w/ 2 levels "Acceptable PM2.5 AQI & Spec"..
##  $ CBSA_CODE                     : int  NA NA NA NA NA NA NA NA NA NA ...
##  $ CBSA_NAME                     : Factor w/ 14 levels "","Asheville, NC",..: 1 1 1..
##  $ STATE_CODE                    : int  37 37 37 37 37 37 37 37 37 37 ...
##  $ STATE                         : Factor w/ 1 level "North Carolina": 1 1 1 1 1 1 ..
##  $ COUNTY_CODE                   : int  11 11 11 11 11 11 11 11 11 11 ...
##  $ COUNTY                        : Factor w/ 21 levels "Avery","Buncombe",..: 1 1 1..
##  $ SITE_LATITUDE                 : num  36 36 36 36 36 ...
##  $ SITE_LONGITUDE                : num  -81.9 -81.9 -81.9 -81.9 -81.9 ...
\end{verbatim}

\begin{Shaded}
\begin{Highlighting}[]
\FunctionTok{str}\NormalTok{(epa\_air\_PM2}\FloatTok{.5}\NormalTok{\_2019, }\AttributeTok{strict.width=}\StringTok{"cut"}\NormalTok{,}\AttributeTok{width=}\DecValTok{85}\NormalTok{)}
\end{Highlighting}
\end{Shaded}

\begin{verbatim}
## 'data.frame':    8581 obs. of  20 variables:
##  $ Date                          : Factor w/ 365 levels "01/01/2019","01/02/2019",...
##  $ Source                        : Factor w/ 2 levels "AirNow","AQS": 2 2 2 2 2 2 2..
##  $ Site.ID                       : int  370110002 370110002 370110002 370110002 370..
##  $ POC                           : int  1 1 1 1 1 1 1 1 1 1 ...
##  $ Daily.Mean.PM2.5.Concentration: num  1.6 1 1.3 6.3 2.6 1.2 1.5 1.5 3.7 1.6 ...
##  $ UNITS                         : Factor w/ 1 level "ug/m3 LC": 1 1 1 1 1 1 1 1 1 ..
##  $ DAILY_AQI_VALUE               : int  7 4 5 26 11 5 6 6 15 7 ...
##  $ Site.Name                     : Factor w/ 25 levels "","Board Of Ed. Bldg.",..: ..
##  $ DAILY_OBS_COUNT               : int  1 1 1 1 1 1 1 1 1 1 ...
##  $ PERCENT_COMPLETE              : num  100 100 100 100 100 100 100 100 100 100 ...
##  $ AQS_PARAMETER_CODE            : int  88502 88502 88502 88502 88502 88502 88502 8..
##  $ AQS_PARAMETER_DESC            : Factor w/ 2 levels "Acceptable PM2.5 AQI & Spec"..
##  $ CBSA_CODE                     : int  NA NA NA NA NA NA NA NA NA NA ...
##  $ CBSA_NAME                     : Factor w/ 14 levels "","Asheville, NC",..: 1 1 1..
##  $ STATE_CODE                    : int  37 37 37 37 37 37 37 37 37 37 ...
##  $ STATE                         : Factor w/ 1 level "North Carolina": 1 1 1 1 1 1 ..
##  $ COUNTY_CODE                   : int  11 11 11 11 11 11 11 11 11 11 ...
##  $ COUNTY                        : Factor w/ 21 levels "Avery","Buncombe",..: 1 1 1..
##  $ SITE_LATITUDE                 : num  36 36 36 36 36 ...
##  $ SITE_LONGITUDE                : num  -81.9 -81.9 -81.9 -81.9 -81.9 ...
\end{verbatim}

\hypertarget{wrangle-individual-datasets-to-create-processed-files.}{%
\subsection{Wrangle individual datasets to create processed
files.}\label{wrangle-individual-datasets-to-create-processed-files.}}

\begin{enumerate}
\def\labelenumi{\arabic{enumi}.}
\setcounter{enumi}{2}
\tightlist
\item
  Change date to date
\item
  Select the following columns: Date, DAILY\_AQI\_VALUE, Site.Name,
  AQS\_PARAMETER\_DESC, COUNTY, SITE\_LATITUDE, SITE\_LONGITUDE
\item
  For the PM2.5 datasets, fill all cells in AQS\_PARAMETER\_DESC with
  ``PM2.5'' (all cells in this column should be identical).
\item
  Save all four processed datasets in the Processed folder. Use the same
  file names as the raw files but replace ``raw'' with ``processed''.
\end{enumerate}

\begin{Shaded}
\begin{Highlighting}[]
\CommentTok{\#3 Change Date from factor variable to date variable with as.Date}
\FunctionTok{class}\NormalTok{(epa\_air\_2018}\SpecialCharTok{$}\NormalTok{Date)}
\end{Highlighting}
\end{Shaded}

\begin{verbatim}
## [1] "factor"
\end{verbatim}

\begin{Shaded}
\begin{Highlighting}[]
\NormalTok{epa\_air\_2018}\SpecialCharTok{$}\NormalTok{Date}\OtherTok{\textless{}{-}}\FunctionTok{as.Date}\NormalTok{(epa\_air\_2018}\SpecialCharTok{$}\NormalTok{Date, }\AttributeTok{format=}\StringTok{"\%m/\%d/\%Y"}\NormalTok{)}
\FunctionTok{class}\NormalTok{(epa\_air\_2018}\SpecialCharTok{$}\NormalTok{Date) }\CommentTok{\#now a date}
\end{Highlighting}
\end{Shaded}

\begin{verbatim}
## [1] "Date"
\end{verbatim}

\begin{Shaded}
\begin{Highlighting}[]
\CommentTok{\#}
\FunctionTok{class}\NormalTok{(epa\_air\_2019}\SpecialCharTok{$}\NormalTok{Date)}
\end{Highlighting}
\end{Shaded}

\begin{verbatim}
## [1] "factor"
\end{verbatim}

\begin{Shaded}
\begin{Highlighting}[]
\NormalTok{epa\_air\_2019}\SpecialCharTok{$}\NormalTok{Date}\OtherTok{\textless{}{-}}\FunctionTok{as.Date}\NormalTok{(epa\_air\_2019}\SpecialCharTok{$}\NormalTok{Date, }\AttributeTok{format=}\StringTok{"\%m/\%d/\%Y"}\NormalTok{)}
\FunctionTok{class}\NormalTok{(epa\_air\_2019}\SpecialCharTok{$}\NormalTok{Date)}\CommentTok{\#now a date}
\end{Highlighting}
\end{Shaded}

\begin{verbatim}
## [1] "Date"
\end{verbatim}

\begin{Shaded}
\begin{Highlighting}[]
\CommentTok{\#}
\FunctionTok{class}\NormalTok{(epa\_air\_PM2}\FloatTok{.5}\NormalTok{\_2018}\SpecialCharTok{$}\NormalTok{Date)}
\end{Highlighting}
\end{Shaded}

\begin{verbatim}
## [1] "factor"
\end{verbatim}

\begin{Shaded}
\begin{Highlighting}[]
\NormalTok{epa\_air\_PM2}\FloatTok{.5}\NormalTok{\_2018}\SpecialCharTok{$}\NormalTok{Date}\OtherTok{\textless{}{-}}\FunctionTok{as.Date}\NormalTok{(epa\_air\_PM2}\FloatTok{.5}\NormalTok{\_2018}\SpecialCharTok{$}\NormalTok{Date, }\AttributeTok{format=}\StringTok{"\%m/\%d/\%Y"}\NormalTok{)}
\FunctionTok{class}\NormalTok{(epa\_air\_PM2}\FloatTok{.5}\NormalTok{\_2018}\SpecialCharTok{$}\NormalTok{Date)}\CommentTok{\#now a date}
\end{Highlighting}
\end{Shaded}

\begin{verbatim}
## [1] "Date"
\end{verbatim}

\begin{Shaded}
\begin{Highlighting}[]
\CommentTok{\#}
\FunctionTok{class}\NormalTok{(epa\_air\_PM2}\FloatTok{.5}\NormalTok{\_2019}\SpecialCharTok{$}\NormalTok{Date)}
\end{Highlighting}
\end{Shaded}

\begin{verbatim}
## [1] "factor"
\end{verbatim}

\begin{Shaded}
\begin{Highlighting}[]
\NormalTok{epa\_air\_PM2}\FloatTok{.5}\NormalTok{\_2019}\SpecialCharTok{$}\NormalTok{Date}\OtherTok{\textless{}{-}}\FunctionTok{as.Date}\NormalTok{(epa\_air\_PM2}\FloatTok{.5}\NormalTok{\_2019}\SpecialCharTok{$}\NormalTok{Date, }\AttributeTok{format=}\StringTok{"\%m/\%d/\%Y"}\NormalTok{)}
\FunctionTok{class}\NormalTok{(epa\_air\_PM2}\FloatTok{.5}\NormalTok{\_2019}\SpecialCharTok{$}\NormalTok{Date)}\CommentTok{\#now a date}
\end{Highlighting}
\end{Shaded}

\begin{verbatim}
## [1] "Date"
\end{verbatim}

\begin{Shaded}
\begin{Highlighting}[]
\DocumentationTok{\#\#\#\#\#\#\#\#\#\#\#\#\#\#\#\#\#}
\DocumentationTok{\#\#\#\#\#\#\#\#\#\#\#\#\#\#\#\#\#}
\CommentTok{\#4 Select the following columns: Date, DAILY\_AQI\_VALUE,}
\CommentTok{\#  Site.Name, AQS\_PARAMETER\_DESC, COUNTY, SITE\_LATITUDE, SITE\_LONGITUDE}
\NormalTok{epa\_air\_2018subset}\OtherTok{\textless{}{-}}\FunctionTok{select}\NormalTok{(epa\_air\_2018,Date, DAILY\_AQI\_VALUE, }
\NormalTok{                           Site.Name, AQS\_PARAMETER\_DESC, COUNTY}\SpecialCharTok{:}\NormalTok{SITE\_LONGITUDE)}
\NormalTok{epa\_air\_2019subset}\OtherTok{\textless{}{-}}\FunctionTok{select}\NormalTok{(epa\_air\_2019,Date, DAILY\_AQI\_VALUE, Site.Name, }
\NormalTok{                           AQS\_PARAMETER\_DESC, COUNTY}\SpecialCharTok{:}\NormalTok{SITE\_LONGITUDE)}
\NormalTok{epa\_air\_PM2}\FloatTok{.5}\NormalTok{\_2018subset}\OtherTok{\textless{}{-}}\FunctionTok{select}\NormalTok{(epa\_air\_PM2}\FloatTok{.5}\NormalTok{\_2018,Date, DAILY\_AQI\_VALUE, }
\NormalTok{                                 Site.Name, AQS\_PARAMETER\_DESC, COUNTY}\SpecialCharTok{:}\NormalTok{SITE\_LONGITUDE)}
\NormalTok{epa\_air\_PM2}\FloatTok{.5}\NormalTok{\_2019subset}\OtherTok{\textless{}{-}}\FunctionTok{select}\NormalTok{(epa\_air\_PM2}\FloatTok{.5}\NormalTok{\_2019,Date, DAILY\_AQI\_VALUE, }
\NormalTok{                                 Site.Name, AQS\_PARAMETER\_DESC, COUNTY}\SpecialCharTok{:}\NormalTok{SITE\_LONGITUDE)}

\DocumentationTok{\#\#\#\#\#\#\#\#\#\#\#\#\#\#\#\#\#}
\DocumentationTok{\#\#\#\#\#\#\#\#\#\#\#\#\#\#\#\#\#}
\CommentTok{\#5 For the PM2.5 datasets, fill all cells in AQS\_PARAMETER\_DESC with "PM2.5"}
\NormalTok{epa\_air\_PM2}\FloatTok{.5}\NormalTok{\_2018subset}\SpecialCharTok{$}\NormalTok{AQS\_PARAMETER\_DESC}\OtherTok{\textless{}{-}}\StringTok{"PM2.5"}
\NormalTok{epa\_air\_PM2}\FloatTok{.5}\NormalTok{\_2019subset}\SpecialCharTok{$}\NormalTok{AQS\_PARAMETER\_DESC}\OtherTok{\textless{}{-}}\StringTok{"PM2.5"}

\CommentTok{\#changed the subsets to factors }
\NormalTok{epa\_air\_PM2}\FloatTok{.5}\NormalTok{\_2018subset}\SpecialCharTok{$}\NormalTok{AQS\_PARAMETER\_DESC}\OtherTok{\textless{}{-}}
  \FunctionTok{as.factor}\NormalTok{(epa\_air\_PM2}\FloatTok{.5}\NormalTok{\_2018subset}\SpecialCharTok{$}\NormalTok{AQS\_PARAMETER\_DESC)}
\NormalTok{epa\_air\_PM2}\FloatTok{.5}\NormalTok{\_2019subset}\SpecialCharTok{$}\NormalTok{AQS\_PARAMETER\_DESC}\OtherTok{\textless{}{-}}
  \FunctionTok{as.factor}\NormalTok{(epa\_air\_PM2}\FloatTok{.5}\NormalTok{\_2019subset}\SpecialCharTok{$}\NormalTok{AQS\_PARAMETER\_DESC)}
\FunctionTok{class}\NormalTok{(epa\_air\_PM2}\FloatTok{.5}\NormalTok{\_2018subset}\SpecialCharTok{$}\NormalTok{AQS\_PARAMETER\_DESC)}
\end{Highlighting}
\end{Shaded}

\begin{verbatim}
## [1] "factor"
\end{verbatim}

\begin{Shaded}
\begin{Highlighting}[]
\FunctionTok{class}\NormalTok{(epa\_air\_PM2}\FloatTok{.5}\NormalTok{\_2019subset}\SpecialCharTok{$}\NormalTok{AQS\_PARAMETER\_DESC)}
\end{Highlighting}
\end{Shaded}

\begin{verbatim}
## [1] "factor"
\end{verbatim}

\begin{Shaded}
\begin{Highlighting}[]
\DocumentationTok{\#\#\#\#\#\#\#\#\#\#\#\#\#\#\#\#\#}
\DocumentationTok{\#\#\#\#\#\#\#\#\#\#\#\#\#\#\#\#\#}
\CommentTok{\#6 Save all four processed datasets in the Processed folder}
\CommentTok{\#2018 EPA air data}
\FunctionTok{write.csv}\NormalTok{(epa\_air\_2018subset, }\AttributeTok{row.names =} \ConstantTok{FALSE}\NormalTok{, }
          \AttributeTok{file=} \StringTok{"Data/Processed/EPAair\_O3\_NC2018\_Processed.csv"}\NormalTok{)}

\CommentTok{\#2019 EPA air data}
\FunctionTok{write.csv}\NormalTok{(epa\_air\_2019subset, }\AttributeTok{row.names=} \ConstantTok{FALSE}\NormalTok{, }
          \AttributeTok{file =} \StringTok{"Data/Processed/EPAair\_O3\_NC2019\_Processed.csv"}\NormalTok{)}

\CommentTok{\#2018 EPA PM2.5 data}
\FunctionTok{write.csv}\NormalTok{(epa\_air\_PM2}\FloatTok{.5}\NormalTok{\_2018subset, }\AttributeTok{row.names=}\ConstantTok{FALSE}\NormalTok{,}
          \AttributeTok{file=}\StringTok{"Data/Processed/EPAair\_PM25\_NC2018\_Processed.csv"}\NormalTok{)}

\CommentTok{\#2019 EPA PM2.5 data}
\FunctionTok{write.csv}\NormalTok{(epa\_air\_PM2}\FloatTok{.5}\NormalTok{\_2019subset, }\AttributeTok{row.names=}\ConstantTok{FALSE}\NormalTok{, }
          \AttributeTok{file=}\StringTok{"Data/Processed/EPAair\_PM25\_NC2019\_Processed.csv"}\NormalTok{)}
\end{Highlighting}
\end{Shaded}

\hypertarget{combine-datasets}{%
\subsection{Combine datasets}\label{combine-datasets}}

\begin{enumerate}
\def\labelenumi{\arabic{enumi}.}
\setcounter{enumi}{6}
\tightlist
\item
  Combine the four datasets with \texttt{rbind}. Make sure your column
  names are identical prior to running this code.
\item
  Wrangle your new dataset with a pipe function (\%\textgreater\%) so
  that it fills the following conditions:
\end{enumerate}

\begin{itemize}
\tightlist
\item
  Include all sites that the four data frames have in common: ``Linville
  Falls'', ``Durham Armory'', ``Leggett'', ``Hattie Avenue'', ``Clemmons
  Middle'', ``Mendenhall School'', ``Frying Pan Mountain'', ``West
  Johnston Co.'', ``Garinger High School'', ``Castle Hayne'', ``Pitt
  Agri. Center'', ``Bryson City'', ``Millbrook School'' (the function
  \texttt{intersect} can figure out common factor levels)
\item
  Some sites have multiple measurements per day. Use the
  split-apply-combine strategy to generate daily means: group by date,
  site, aqs parameter, and county. Take the mean of the AQI value,
  latitude, and longitude.
\item
  Add columns for ``Month'' and ``Year'' by parsing your ``Date'' column
  (hint: \texttt{lubridate} package)
\item
  Hint: the dimensions of this dataset should be 14,752 x 9.
\end{itemize}

\begin{enumerate}
\def\labelenumi{\arabic{enumi}.}
\setcounter{enumi}{8}
\tightlist
\item
  Spread your datasets such that AQI values for ozone and PM2.5 are in
  separate columns. Each location on a specific date should now occupy
  only one row.
\item
  Call up the dimensions of your new tidy dataset.
\item
  Save your processed dataset with the following file name:
  ``EPAair\_O3\_PM25\_NC1718\_Processed.csv''
\end{enumerate}

\begin{Shaded}
\begin{Highlighting}[]
\CommentTok{\#7 Combine the four datasets with \textasciigrave{}rbind\textasciigrave{}}
\CommentTok{\#Import processed EPA air data}
\NormalTok{epa2018}\OtherTok{\textless{}{-}}\FunctionTok{read.csv}\NormalTok{(}\StringTok{"Data/Processed/EPAair\_O3\_NC2018\_Processed.csv"}\NormalTok{,}
                  \AttributeTok{stringsAsFactors =} \ConstantTok{TRUE}\NormalTok{)}
\NormalTok{epa2019}\OtherTok{\textless{}{-}}\FunctionTok{read.csv}\NormalTok{(}\StringTok{"Data/Processed/EPAair\_O3\_NC2019\_Processed.csv"}\NormalTok{,}
                  \AttributeTok{stringsAsFactors =} \ConstantTok{TRUE}\NormalTok{)}
\NormalTok{epa2018\_pm25}\OtherTok{\textless{}{-}}\FunctionTok{read.csv}\NormalTok{(}\StringTok{"Data/Processed/EPAair\_PM25\_NC2018\_Processed.csv"}\NormalTok{,}
                       \AttributeTok{stringsAsFactors =} \ConstantTok{TRUE}\NormalTok{)}
\NormalTok{epa2019\_pm25}\OtherTok{\textless{}{-}}\FunctionTok{read.csv}\NormalTok{(}\StringTok{"Data/Processed/EPAair\_PM25\_NC2019\_Processed.csv"}\NormalTok{,}
                       \AttributeTok{stringsAsFactors =} \ConstantTok{TRUE}\NormalTok{)}

\CommentTok{\#rbind()}
\NormalTok{EPA\_air\_2018thru2019}\OtherTok{\textless{}{-}}\FunctionTok{rbind}\NormalTok{(epa2018, epa2018\_pm25, }
\NormalTok{                            epa2019, epa2019\_pm25)}
\FunctionTok{class}\NormalTok{(EPA\_air\_2018thru2019}\SpecialCharTok{$}\NormalTok{Date) }\CommentTok{\#class is originally factor}
\end{Highlighting}
\end{Shaded}

\begin{verbatim}
## [1] "factor"
\end{verbatim}

\begin{Shaded}
\begin{Highlighting}[]
\CommentTok{\#change class to date with as.Date()}
\NormalTok{EPA\_air\_2018thru2019}\SpecialCharTok{$}\NormalTok{Date}\OtherTok{\textless{}{-}}
  \FunctionTok{as.Date}\NormalTok{(EPA\_air\_2018thru2019}\SpecialCharTok{$}\NormalTok{Date,}\AttributeTok{format=}\StringTok{"\%Y{-}\%m{-}\%d"}\NormalTok{)}
\FunctionTok{class}\NormalTok{(EPA\_air\_2018thru2019}\SpecialCharTok{$}\NormalTok{Date) }\CommentTok{\#Now class is date}
\end{Highlighting}
\end{Shaded}

\begin{verbatim}
## [1] "Date"
\end{verbatim}

\begin{Shaded}
\begin{Highlighting}[]
\CommentTok{\#}
\FunctionTok{class}\NormalTok{(EPA\_air\_2018thru2019}\SpecialCharTok{$}\NormalTok{DAILY\_AQI\_VALUE) }\CommentTok{\#class:integer }
\end{Highlighting}
\end{Shaded}

\begin{verbatim}
## [1] "integer"
\end{verbatim}

\begin{Shaded}
\begin{Highlighting}[]
\FunctionTok{class}\NormalTok{(EPA\_air\_2018thru2019}\SpecialCharTok{$}\NormalTok{SITE\_LATITUDE) }\CommentTok{\#class:numeric}
\end{Highlighting}
\end{Shaded}

\begin{verbatim}
## [1] "numeric"
\end{verbatim}

\begin{Shaded}
\begin{Highlighting}[]
\FunctionTok{class}\NormalTok{(EPA\_air\_2018thru2019}\SpecialCharTok{$}\NormalTok{SITE\_LONGITUDE) }\CommentTok{\#class: numeric}
\end{Highlighting}
\end{Shaded}

\begin{verbatim}
## [1] "numeric"
\end{verbatim}

\begin{Shaded}
\begin{Highlighting}[]
\DocumentationTok{\#\#\#\#\#\#\#\#\#\#\#\#\#\#\#\#\#\#\#\#\#\#\#\#\#\#\#\#\#\#\#\#\#\#\#\#\#\#\#\#\#\#\#\#\#\#\#\#\#\#\#\#\#\#\#}
\CommentTok{\#8 Wrangle new dataset with pipeline}
\NormalTok{EPA\_air\_total\_processed }\OtherTok{\textless{}{-}}\NormalTok{ EPA\_air\_2018thru2019 }\SpecialCharTok{\%\textgreater{}\%} 
  \FunctionTok{filter}\NormalTok{(Site.Name }\SpecialCharTok{\%in\%} \FunctionTok{c}\NormalTok{(}\StringTok{"Linville Falls"}\NormalTok{, }\StringTok{"Durham Armory"}\NormalTok{, }
                                  \StringTok{"Leggett"}\NormalTok{, }\StringTok{"Hattie Avenue"}\NormalTok{, }\StringTok{"Clemmons Middle"}\NormalTok{, }
                                \StringTok{"Mendenhall School"}\NormalTok{,}\StringTok{"Frying Pan Mountain"}\NormalTok{, }
                                \StringTok{"West Johnston Co."}\NormalTok{, }\StringTok{"Garinger High School"}\NormalTok{, }
                                \StringTok{"Castle Hayne"}\NormalTok{, }\StringTok{"Pitt Agri. Center"}\NormalTok{, }\StringTok{"Bryson City"}\NormalTok{, }
                                \StringTok{"Millbrook School"}\NormalTok{)) }\SpecialCharTok{\%\textgreater{}\%}
  \FunctionTok{group\_by}\NormalTok{(Date, Site.Name, AQS\_PARAMETER\_DESC, COUNTY) }\SpecialCharTok{\%\textgreater{}\%}
  \FunctionTok{summarise}\NormalTok{(}\AttributeTok{mean\_dailyAQI=}\FunctionTok{mean}\NormalTok{(DAILY\_AQI\_VALUE),}
            \AttributeTok{mean\_Latitude=}\FunctionTok{mean}\NormalTok{(SITE\_LATITUDE),}
            \AttributeTok{mean\_Longitude=}\FunctionTok{mean}\NormalTok{(SITE\_LONGITUDE)) }\SpecialCharTok{\%\textgreater{}\%}
  \FunctionTok{mutate}\NormalTok{(}\AttributeTok{Month=}\FunctionTok{month}\NormalTok{(Date), }\AttributeTok{Year=}\FunctionTok{year}\NormalTok{(Date))}
\end{Highlighting}
\end{Shaded}

\begin{verbatim}
## `summarise()` has grouped output by 'Date', 'Site.Name', 'AQS_PARAMETER_DESC'. You can override using the `.groups` argument.
\end{verbatim}

\begin{Shaded}
\begin{Highlighting}[]
\FunctionTok{dim}\NormalTok{(EPA\_air\_total\_processed) }\CommentTok{\#now the dimensions are 14752 x 9 }
\end{Highlighting}
\end{Shaded}

\begin{verbatim}
## [1] 14752     9
\end{verbatim}

\begin{Shaded}
\begin{Highlighting}[]
\CommentTok{\#9}
\NormalTok{EPA\_air\_processed\_spread}\OtherTok{\textless{}{-}} \FunctionTok{pivot\_wider}\NormalTok{(EPA\_air\_total\_processed, }
                  \AttributeTok{names\_from =}\NormalTok{ AQS\_PARAMETER\_DESC, }\AttributeTok{values\_from =}\NormalTok{ mean\_dailyAQI)}

\CommentTok{\#10 Call up the dimensions of your new tidy dataset.}
\FunctionTok{dim}\NormalTok{(EPA\_air\_processed\_spread) }\CommentTok{\#8976 x 9}
\end{Highlighting}
\end{Shaded}

\begin{verbatim}
## [1] 8976    9
\end{verbatim}

\begin{Shaded}
\begin{Highlighting}[]
\CommentTok{\#11 Save your processed dataset. }
\CommentTok{\#Instead of NC1718, I named it NC1819 to fit the years of the data}
\FunctionTok{write.csv}\NormalTok{(EPA\_air\_processed\_spread, }
          \AttributeTok{row.names=}\ConstantTok{FALSE}\NormalTok{,}\AttributeTok{file=}\StringTok{"Data/Processed/EPAair\_O3\_PM25\_NC1819\_Processed.csv"}\NormalTok{)}
\end{Highlighting}
\end{Shaded}

\hypertarget{generate-summary-tables}{%
\subsection{Generate summary tables}\label{generate-summary-tables}}

\begin{enumerate}
\def\labelenumi{\arabic{enumi}.}
\setcounter{enumi}{11}
\item
  Use the split-apply-combine strategy to generate a summary data frame.
  Data should be grouped by site, month, and year. Generate the mean AQI
  values for ozone and PM2.5 for each group. Then, add a pipe to remove
  instances where a month and year are not available (use the function
  \texttt{drop\_na} in your pipe).
\item
  Call up the dimensions of the summary dataset.
\end{enumerate}

\begin{Shaded}
\begin{Highlighting}[]
\CommentTok{\#12a}
\CommentTok{\# I just used the dataframe where I spread the data: EPA\_air\_processed\_spread}
\NormalTok{EPA\_air\_meanAQI\_summary }\OtherTok{\textless{}{-}}\NormalTok{ EPA\_air\_processed\_spread }\SpecialCharTok{\%\textgreater{}\%}
  \FunctionTok{group\_by}\NormalTok{( Site.Name, Month,Year) }\SpecialCharTok{\%\textgreater{}\%}
  \FunctionTok{summarise}\NormalTok{(}\AttributeTok{mean\_AQI\_ozone =}\FunctionTok{mean}\NormalTok{(Ozone),}
            \AttributeTok{mean\_AQI\_PM2.5 =}\FunctionTok{mean}\NormalTok{(PM2}\FloatTok{.5}\NormalTok{))}
\end{Highlighting}
\end{Shaded}

\begin{verbatim}
## `summarise()` has grouped output by 'Site.Name', 'Month'. You can override using the `.groups` argument.
\end{verbatim}

\begin{Shaded}
\begin{Highlighting}[]
\CommentTok{\#12b \#I used drop\_na for any mean AQI values that did not exist for the specified months and years}
\NormalTok{EPA\_air\_meanAQI\_summary\_cleaned }\OtherTok{\textless{}{-}}\NormalTok{ EPA\_air\_meanAQI\_summary }\SpecialCharTok{\%\textgreater{}\%}
  \FunctionTok{drop\_na}\NormalTok{(mean\_AQI\_ozone) }\SpecialCharTok{\%\textgreater{}\%}
  \FunctionTok{drop\_na}\NormalTok{(mean\_AQI\_PM2}\FloatTok{.5}\NormalTok{)}

\CommentTok{\#13}
\FunctionTok{dim}\NormalTok{(EPA\_air\_meanAQI\_summary\_cleaned)}
\end{Highlighting}
\end{Shaded}

\begin{verbatim}
## [1] 101   5
\end{verbatim}

\begin{Shaded}
\begin{Highlighting}[]
\CommentTok{\#14}
\CommentTok{\#?drop\_na {-} I read the R description }
\CommentTok{\#?na.omit {-} Iread the R description }
\CommentTok{\#I commented the pipe out to test out the na.omit function}
\CommentTok{\#EPA\_air\_meanAQI\_summary\_clean \textless{}{-} EPA\_air\_meanAQI\_summary \%\textgreater{}\%}
 \CommentTok{\# na.omit(mean\_AQI\_ozone) \#\textgreater{}\%\textgreater{} }
 \CommentTok{\#na.omit(mean\_AQI\_PM2.5)}
\CommentTok{\#when I exclude the pipe, the missing values in PM2.5 also get deleted }
\CommentTok{\#so na.omit doesn\textquotesingle{}t specify removing values by column }
\end{Highlighting}
\end{Shaded}

\begin{enumerate}
\def\labelenumi{\arabic{enumi}.}
\setcounter{enumi}{13}
\tightlist
\item
  Why did we use the function \texttt{drop\_na} rather than
  \texttt{na.omit}?
\end{enumerate}

\begin{quote}
Answer: When I used both functions on the summary data, they provided
the same dimensions. I think we used drop\_na instead of na.omit because
you can specify which rows to remove based on the specified column with
drop\_na, whereas other columns that may have missing values will get
deleted when using na.omit. Since in this dataframe, all the missing
values were in the mean AQI data, na.omit and drop\_na produced the same
dimensions. However, if there was a missing value in Site.Name, it may
recognize that missing value and delete that entire row. So drop\_na
just removes NAs based on the column you want, not other columns as
well. I tried this and deleted the na.omit(mean\_AQI\_PM2.5) pipe from
the pipe I tried for \#14 and found that even just with the
na.omit(mean\_AQI\_ozone), it deleted missing values for both columns.
So na.omit, just deletes all the missing values and cannot specify the
missing values by column like with drop\_na.
\end{quote}

\end{document}
